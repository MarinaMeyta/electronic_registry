\input{config}

\begin{document}

\input{title}
\newpage
\ESKDthisStyle{empty}
\paragraph{\hfill РЕФЕРАТ \hfill}
Курсовая работа содержит \ESKDtotal{page} страниц, \ESKDtotal{figure} рисунков, \ESKDtotal{table} таблиц, \ESKDtotal{bibitem} источника, \ESKDtotal{appendix} приложение.

БАЗЫ ДАННЫХ, SQLITE, MONODEVELOP, C\#, GTKSharp.

Цель работы --- проектирование, разработка базы данных и клиентской части программного обеспечения для электронной регистрации на прием к врачу (электронная регистратура).

Результатом выполнения работы является база данных и графическое приложение для осуществления регистрации пациентов, записи на прием к специалисту, администрирования записей в базе данных (просмотр, удаление и добавление информации о сотрудниках поликлиники, пациентах, выданных талонах и др.).

В процессе работы были выполнены все вышепоставленные цели, разработана инфологическая модель данных для описания процесса регистрации и структуры базы данных, предусмотрены ограничения на ввод данных, а также применены средства обеспечения безопасности базы данных на уровне приложения.

Проект выполнен с использованием следующих средств разработки:
\begin{itemize}
  \item ОС Linux Ubuntu 14.10;
  \item язык программирования C\#;
  \item среда разработки MonoDevelop 4.0.12 \cite{monodoc};
  \item встраиваемая реляционная база данных SQLite \cite{sqlitedoc};
  \item СУБД SQLiteman 1.2.2;
  \item кроссплатформенная библиотека элементов графического интерфейса GTKSharp;
  \item iTextSharp --- .NET PDF библиотека для генерации PDF-документов \cite{itextsharp};
  \item система контроля версий Git.
\end{itemize}

Пояснительная записка выполнена при помощи системы компьютерной вёрстки \LaTeX.


\newpage
\ESKDstyle{plain}
\tableofcontents

\newpage
\ESKDstyle{plain}
\section*{Введение}
\addcontentsline{toc}{section}{Введение}
В качестве задания на курсовую работу была поставлена задача разработать базу данных и программу пользователя для осуществления электронной регистрации (записи на прием к врачу) в поликлинике.  


\section{Проектирование инфологической модели данных}
Инфологическая (концептуальная) модель предметной области представляет собой информационную модель наиболее высокого уровня абстракции и в сущности является как образом реальности, так и образом проектируемой базы данных для этой реальности. Она включает в себя описание информационных объектов или понятий предметной области и связей между ними. а также описание ограничений целостности, т.е. требований к допустимым значениям данных и к связям между ними.

Описание бизнесс-процессов в системе электронной регистрации пациентов представлено на диаграммах IDEF0, DFD IDEF3 (рисунки \ref{idef0_1:idef0_1}-\ref{idef3_2:idef3_2}).

\begin{figure}[h!]
\center{\includegraphics[width=0.5\linewidth]{idef0_1}}
\caption{<<Черный ящик>>}
\label{idef0_1:idef0_1}
\end{figure} 

\begin{figure}[h!]
\center{\includegraphics[width=0.9\linewidth]{idef0_2}}
\caption{Диаграмма IDEF0}
\label{idef0_2:idef0_2}
\end{figure} 

\begin{figure}[h!]
\center{\includegraphics[width=0.9\linewidth]{dfd}}
\caption{DFD-диаграмма бизнес-процессов}
\label{dfd:dfd}
\end{figure} 

\begin{figure}[h!]
\center{\includegraphics[width=0.9\linewidth]{idef3_1}}
\caption{Диаграмма IDEF3 (часть 1)}
\label{idef3_1:idef3_1}
\end{figure} 

\begin{figure}[h!]
\center{\includegraphics[width=0.4\linewidth]{idef3_2}}
\caption{Диаграмма IDEF3 (часть 2)}
\label{idef3_2:idef3_2}
\end{figure} 

\clearpage






\section{Описание базы данных}
\setcounter{figure}{0}

\subsection{Таблица <<patient>>}
Ограничения на таблицу <<patient>> (пациент):

\begin{itemize}
  \item ID пациента не меньше единицы;
  \item фамилия, имя отчество не должно превосходить 50 символов;
  \item ни один из атрибутов не дожен быть пустым (NULL).
\end{itemize}

\subsection{Таблица <<passport>>}
Ограничения на таблицу <<passport>> (паспорт):

\begin{itemize}
  \item ID пасспорта не меньше единицы;
  \item серия паспорта не должна превосходить 4 символа;
  \item номер паспорта не должен превосходить 6 символов;
  \item адрес места жительства не должен превосходить 100 символов;
  \item атрибут <<пол>> должен состоять из 1 символа (М/Ж);
  \item ни один из атрибутов не дожен быть пустым (NULL).
\end{itemize}

\subsection{Таблица <<policy>>}
Ограничения на таблицу <<policy>> (полис):

\begin{itemize}
  \item ID полиса не меньше единицы;
  \item название страховой медицинской компании не должно превосходить 100 символов;
  \item ни один из атрибутов не дожен быть пустым (NULL).
\end{itemize}

\subsection{Таблица <<talon>>}
Ограничения на таблицу <<talon>> (талон):

\begin{itemize}
  \item ID талона не меньше единицы;
  \item ни один из атрибутов не дожен быть пустым (NULL).
\end{itemize}

\subsection{Таблица <<timetable>>}
Ограничения на таблицу <<timetable>> (расписание):

\begin{itemize}
  \item ID расписания не меньше единицы;
  \item день недели должен состоять из 2-ух символов (<<Пн>>, <<Вт>> и т.д.);
  \item ни один из атрибутов не дожен быть пустым (NULL).
\end{itemize}

\subsection{Таблица <<employee>>}
Ограничения на таблицу <<employee>> (сотрудник):

\begin{itemize}
  \item ID сотрудника не меньше единицы;
  \item специальность сотрудника не должна превосходить 50 символов;
  \item фамилия, имя отчество не должно превосходить 50 символов;
  \item номер рабочего кабинета должен состоять из 3 символов;
  \item ни один из атрибутов не дожен быть пустым (NULL).
\end{itemize}



\section{Описание процесса деятельности} 
\setcounter{figure}{0}

\subsection{Постановка задачи}
\input{problem_form}
\subsection{Описание данных программы}
\input{data_description}
\subsection{Основные технические решения}
\subsubsection{Алгоритм}

Предварительного заполнения базы данных не требуется, однако для возможности добавления записи и регистрации пациента администратору сначала необходимо заполнить базу сотрудников и расписаний.

\begin{center}
  Шаг 1
\end{center}

При авторизации от имени пациента в главном окне программы (MainWindow) появится окно для записи на прием к врачу (EnrollWindow). Необходимо выбрать нужного специалиста, найти его расписание и правильно заполнить предлагаемую форму. Если входные значения были введены неверно, появится сообщение об ошибке. 

\begin{center}
  Шаг 2
\end{center}

Программа проверяет, есть ли уже в БД введенные пользователем серия и номер паспорта. Если данные о пациенте в базе уже есть и форма заполнена верно, появится сообщение об успешной записи, либо сообщение об ошибке записи, если такая запись уже была добавлена в БД.

\begin{center}
  Шаг 3
\end{center}

Если данных о пациенте в базе нет, то появится форма регистрации пациента (PatientRegisterWindow). Если все поля формы заполнены верно, пациент будет добавлен в базу и в окне записи на прием при повторной операции записи появится уведомление об успешном завершении операции, в результате которой будет сгенерирован PDF-файл (талон), готовый к печати.

\begin{center}
  Шаг 4
\end{center}

При авторизации от имени администратора программа сверяет хеш введённого пароля с  хешем, который хранится в БД. Если они совпадают, то открывается окно управления базой данных (AdminWindow), если нет, то появляется уведомление об ошибочной авторизации.

\begin{center}
  Шаг 5
\end{center}

Окно управления базой данных (AdminWindow) позволяет администратору просматривать, удалять и добавлять данные в базу. Для этого необходимо вводить корректные данные, в противном случае операции удаления и добавления выполняться не будут, появится сообщение об ошибке ввода или обращения к базе.

Блок-схема алгоритма работы программы представлена на рисунке~\ref{block:block}.

\begin{figure}[h!]
\center{\includegraphics[width=0.5\linewidth]{block}}
\caption{Блок-схема алгоритма работы программы}
\label{block:block}
\end{figure}

\subsubsection{Численность, функции и квалификация персонала}

Для использования системы необходим администратор, который будет добавлять новых сотрудников и менять расписание уже добавленных, следить за целостностью системы, удалять ПДн пациентов по мере необходимости.

\subsubsection{Обеспечение потребительских характеристик системы}

Надежность обеспечивается путем следования стандартам написания кода и использования блоков try-catch для обработки исключительных ситуаций.
Производительность системы обеспечивается путем использования оптимальных алгоритмов.

\subsubsection{Функции, выполняемые системой}

Функциями, выполняемыми программой <<hospital\_register>>, являются регистрация пациентов, добавление записей на прием, печать талонов, управление расписанием и номенклатурой сотрудников (врачей), а также управление записями в БД. 

\subsubsection{Комплекс технических средств}

Для функционирования системы необходимы следующие аппаратные средства:

\begin{itemize}
  \item ОС Linux Ubuntu/Debian;
  \item процессор x86-архитектуры;
  \item объем ОЗУ для выполнения программы: не менее 300Мб;
  \item объём видеопамяти: не менее 300Мб;
  \item память на жестком диске: не менее 100Мб для файлов БД и программы;
  \item монитор с разрешением 800x600 или выше;
  \item мышь, клавиатура;
  \item устройство для чтения CD;
  \item принтер.
\end{itemize}

\subsubsection{Информационное обеспечение системы}

Система поставляется с руководством пользователя и программиста.

\subsubsection{Программное обеспечение системы}

Система разворачивается на компьютере с ОС Linux Ubuntu/Debian.

\section{Мероприятия по подготовке персонала}

Провести ознакомление персонала с руководством пользователя.








\section{Руководство пользователя}
\setcounter{figure}{0}

\section{Перспективы применения программы}
\setcounter{figure}{0}

\section{Заключение}
\setcounter{figure}{0}

% \section{Инструменты}
% \setcounter{figure}{0}
% \subsection{Система контроля версий Git}
% \input{git}
% \subsection{Система компьютерной вёрстки \TeX}
% \input{tex}
% \section{Технические характеристики}
% \input{technical_attributes}



\newpage
\section*{Заключение}
\addcontentsline{toc}{section}{Заключение}
\textbf{ПРАВИТЬ!!!!!!!!!!!!!!!!!!!!!!!!!!!!!!!!!!!!}

\newpage
\renewcommand{\refname}{Список использованных источников}
\bibliography{lit}

\ESKDappendix{Обязательное}{\normalfont Компакт-диск}
Компакт-диск содержит: 
\begin{itemize}
\item электронную версию пояснительной записки в форматах *.tex и *.pdf;
\item актуальную версию программного комплекса для проведения компьютерной экспертизы;
\item базу данных.
\end{itemize}

\end{document}
