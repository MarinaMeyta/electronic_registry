\subsection{Постановка задачи}

База данных и программа <<hospital\_register>> создается для внедрения в поликлиниках в качестве электронной регистратуры.

\subsection{Описание данных программы}
\subsubsection{Входные данные}

В программе <<hospital\_register>> возможны следующие входные данные:
Ф.И.О. сотрудника, специальность сотрудника, день записи, серия и номер паспорта пациента, номер кабинета, день недели, ID сотрудника, начало и конец смены сотрудника, ID пациента, ID расписания и ID талона, Ф.И.О. пациента, день рождения пациента, месяц рождения пациента, год рождения пациента, пол пациента, код подразделения, адрес места жительства пациента, серия полиса пациента, номер полиса пациента, страховая компания.
Далее приведен диапазон допустимых значений. Все входные значения не должны быть пустыми (NULL).

В главном окне программы MainWindow возможен только один входной параметр --- пароль администратора. На него заведомо накладывается только одно ограничение --- пароль не должен быть пустой строкой.

В окне управления базой AdminWindow входные параметры следующие: Ф.И.О. сотрудника, специальность сотрудника, номер кабинета, день недели, ID сотрудника, начало и конец смены сотрудника, ID пациента, ID расписания и ID талона (таблица~\ref{tab:tab_1}). 

\begin{table}[ht]
\caption{Таблица признаков для AdminWindow}
\label{tab:tab_1}
\begin{center}
\begin{tabularx}{\linewidth}{|X|X|X|}
\hline
 Наименование признака & Описание признака & Диапазон допустимых значений\\
\hline
 employee\_name & Ф.И.О. сотрудника & Символы русского или латинского алфавита\\
\hline
 speciality & Специальность сотрудника & Символы русского или латинского алфавита\\
\hline
 office\_number & Номер кабинета сотрудника & 3 цифры от 0 до 9\\
\hline
 week\_day & День недели & Одно из заранее определенных значений (<<Пн>>, <<Вт>>, <<Ср>>, <<Чт>>, <<Пт>>, <<Сб>>, <<Вс>>)\\
\hline
 employee\_id & Идентификатор сотрудника & 10 цифр от 0 до 9 \\
\hline
 patient\_id & Идентификатор пациента & 10 цифр от 0 до 9 \\
\hline
 timetable\_id & Идентификатор расписания & 10 цифр от 0 до 9 \\
\hline
 talon\_id & Идентификатор талона & 10 цифр от 0 до 9 \\
\hline
 shift\_begining & Начало смены сотрудника & Строка формата <<ЧЧ:ММ>>, где ЧЧ (часы) и ММ (минуты) --- цифры от 0 до 9 \\
\hline
 shift\_ending & Конец смены сотрудника & Строка формата <<ЧЧ:ММ>>, где ЧЧ (часы) и ММ (минуты) --- цифры от 0 до 9\\
\hline
\end{tabularx}
\end{center}
\end{table}

В окне для записи на прием к врачу EnrollWindow входные параметры: Ф.И.О. сотрудника, специальность сотрудника, день записи, серия и номер паспорта пациента (таблица~\ref{tab:tab_2}). 

\begin{table}[ht]
\caption{Таблица признаков для EnrollWindow}
\label{tab:tab_2}
\begin{center}
\begin{tabularx}{\linewidth}{|X|X|X|}
\hline
 Наименование признака & Описание признака & Диапазон допустимых значений\\
\hline
 employee\_name & Ф.И.О. сотрудника & Одно из заранее определенных значений\\
\hline
 speciality & Специальность сотрудника & Одно из заранее определенных значений\\
\hline
 week\_day & День записи & Одно из заранее определенных значений (<<Пн>>, <<Вт>>, <<Ср>>, <<Чт>>, <<Пт>>, <<Сб>>, <<Вс>>)\\
\hline
 passport & Серия пасспорта пациента & 4 цифры от 0 до 9 \\
\hline
 passport & Номер паспорта пациента & 6 цифр от 0 до 9 \\
\hline
\end{tabularx}
\end{center}
\end{table}

В окне регистрации пациента PatientRegisterWindow входные параметры: Ф.И.О. пациента, день рождения пациента, месяц рождения пациента, год рождения пациента, пол пациента, серия паспорта пациента, номер паспорта пациента, код подразделения, адрес места жительства пациента, серия полиса пациента, номер полиса пациента, страховая компания (таблица~\ref{tab:tab_3}). 

\begin{table}[ht]
\caption{Таблица признаков для EnrollWindow}
\label{tab:tab_3}
\begin{center}
\begin{tabularx}{\linewidth}{|X|X|X|}
\hline
 Наименование признака & Описание признака & Диапазон допустимых значений\\
\hline
 patient\_name & Ф.И.О. сотрудника & Символы русского или латинского алфавита\\
\hline
 birth\_date & День рождения пациента & Число от 1 до 31\\
\hline
 birth\_date & Месяц рождения пациента & Одно из заранее определенных значений\\
\hline
 birth\_date & Год рождения пациента & Число от 1900 до 2020\\
\hline
 sex & Пол пациента & Одно из заранее определенных значений (<<М>> или <<Ж>>)\\
\hline
 passport & Серия пасспорта пациента & 4 цифры от 0 до 9 \\
\hline
 passport & Номер паспорта пациента & 6 цифр от 0 до 9 \\
\hline
 issue\_date & День выдачи паспорта & Число от 1 до 31\\
\hline
 issue\_date & Месяц выдачи паспорта & Одно из заранее определенных значений\\
\hline
 issue\_date & Год выдачи паспорта & Число от 1900 до 2020\\
\hline
 policy & Серия полиса пациента & 5 цифр от 0 до 9 \\
\hline
 policy & Номер полиса пациента & 6 цифр от 0 до 9 \\
\hline
 insurance\_agency & Название страховой медицинской компании & Символы русского или латинского алфавита \\
\hline
\end{tabularx}
\end{center}
\end{table}

\clearpagepage

\subsubsection{Выходные данные}

Выходные данные сохраняются в PDF-файл <<talon.pdf>> в рабочей директории программы. Данный файл в дальнейшем может быть распечатан терминалом при завершении операции записи пациента. 

В файл <<talon.pdf>> выводятся следующие данные: Ф.И.О. врача, специальность, номер кабинета, рабочие часы, дата и день недели, Ф.И.О. пациента и страховой медицинский полис пациента.











