\subsection{Постановка задачи}

База данных и программа <<hospital\_register>> создается для внедрения в поликлиниках в качестве электронной регистратуры.

\subsection{Описание данных программы}
\subsubsection{Входные данные}

В программе <<hospital\_register>> возможны следующие входные данные:
\textbf{ПЕРЕЧИСЛИИИИИТЬ!!!!!}
Далее приведен диапазон допустимых значений. Все входные значения не должны быть пустыми (NULL).

В главном окне программы MainWindow возможен только один входной параметр --- пароль администратора. На него заведомо накладывается только одно ограничение --- пароль не должен быть пустой строкой.

В окне управления базой AdminWindow входные параметры следующие: Ф.И.О. сотрудника, специальность сотрудника, номер кабинета, день недели, ID сотрудника, начало и конец смены сотрудника, ID пациента, ID расписания и ID талона (таблица~\ref{tab:tab_1}). 

\begin{table}[ht]
\caption{Таблица признаков для AdminWindow}
\label{tab:tab_1}
\begin{center}
\begin{tabularx}{\linewidth}{|X|X|X|}
\hline
Наименование признака & Описание признака & Диапазон допустимых значений\\
\hline
 employee\_name & Ф.И.О. сотрудника & Символы русского/латинского алфавита\\
\hline
 speciality & Специальность сотрудника & Символы русского/латинского алфавита\\
\hline
 office\_number & Номер кабинета сотрудника & 3 цифры от 0 до 9\\
\hline
 week\_day & День недели & Одно из заранее определенных значений (<<Пн>>, <<Вт>>, <<Ср>>, <<Чт>>, <<Пт>>, <<Сб>>, <<Вс>>)\\
\hline
 employee\_id & Идентификатор сотрудника & 10 цифр от 0 до 9 \\
\hline
 patient\_id & Идентификатор пациента & 10 цифр от 0 до 9 \\
\hline
 timetable\_id & Идентификатор расписания & 10 цифр от 0 до 9 \\
\hline
 talon\_id & Идентификатор талона & 10 цифр от 0 до 9 \\
\hline
 shift\_begining & Начало смены сотрудника & Строка формата <<ЧЧ:ММ>>, где ЧЧ (часы) и ММ (минуты) --- цифры от 0 до 9 \\
\hline
 shift\_ending & Конец смены сотрудника & Строка формата <<ЧЧ:ММ>>, где ЧЧ (часы) и ММ (минуты) --- цифры от 0 до 9\\
\hline
\end{tabularx}
\end{center}
\end{table}

В окне для записи на прием к врачу EnrollWindow входные параметры: Ф.И.О. сотрудника, специальность сотрудника, день записи, серия и номер паспорта пациента (таблица~\ref{tab:tab_2}). 

\begin{table}[ht]
\caption{Таблица признаков для EnrollWindow}
\label{tab:tab_2}
\begin{center}
\begin{tabularx}{\linewidth}{|X|X|X|}
\hline
Наименование признака & Описание признака & Диапазон допустимых значений\\
\hline
 employee\_name & Ф.И.О. сотрудника & Одно из заранее определенных значений\\
\hline
 speciality & Специальность сотрудника & Одно из заранее определенных значений\\
\hline
 week\_day & День записи & Одно из заранее определенных значений (<<Пн>>, <<Вт>>, <<Ср>>, <<Чт>>, <<Пт>>, <<Сб>>, <<Вс>>)\\
\hline
 passport & Серия пасспорта пациента & 4 цифры от 0 до 9 \\
\hline
 passport & Номер паспорта пациента & 6 цифр от 0 до 9 \\
\hline
\end{tabularx}
\end{center}
\end{table}



\subsubsection{Выходные данные}
