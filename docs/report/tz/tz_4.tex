\subsection{Общие требования к системе}
\subsubsection{Входные и выходные данные}

Входные данные:

\input{input}

Выходные данные:

Выходные данные сохраняются в PDF-файл <<talon.pdf>> в рабочей директории программы. Данный файл в дальнейшем может быть распечатан терминалом при завершении операции записи пациента. 

В файл <<talon.pdf>> выводятся следующие данные: Ф.И.О. врача, специальность, номер кабинета, рабочие часы, дата и день недели, Ф.И.О. пациента и страховой медицинский полис пациента.


\clearpage
\subsubsection{Требования к транспортированию и хранению}

Программа поставляется на CD, который должен быть помещен в жесткий футляр, обеспечивающий длительное хранение в отапливаемых помещениях в условиях, пригодных для хранения оптических дисков.
При транспортировании и хранении программного изделия должна быть предусмотрена защита от попадания пыли и атмосферных осадков. Климатические условия транспортирования:

\begin{itemize}
  \item температура окружающего воздуха, °С – от 5 до 50;
  \item атмосферное давление, кПа – 84,0-106,7;
  \item относительная влажность воздуха при 25 °С – 45-80\%.
\end{itemize}

\subsubsection{Требования к информационной и программной совместимости}

Программное обеспечение <<hospital\_register>> с ОС Linux Ubuntu/Debian.

Минимальные системные требования:

\begin{itemize}
  \item ОС Linux Ubuntu/Debian;
  \item процессор x86-архитектуры;
  \item объем ОЗУ для выполнения программы: не менее 300 Мб;
  \item объём видеопамяти: не менее 300 Мб;
  \item память на жестком диске: не менее 100 Мб для файлов БД и программы;
  \item монитор с разрешением 800x600 или выше;
  \item мышь, клавиатура;
  \item устройство для чтения CD;
  \item принтер.
\end{itemize}

\subsubsection{Требования к надежности}

При условии соблюдения требований эксплуатации и при контроле входных и выходных данных программа должна безотказно выполнять функции, определенные настоящим техническим заданием.

Время восстановления после отказа, вызванного сбоем технических средств, сбоем операционной системы, не должно превышать времени, необходимого на перезагрузку операционной системы и запуск программы.

Время восстановления после отказа, вызванного неисправностью технических средств, жестким сбоем операционной системы, не должно превышать времени, требуемого на устранение неисправностей технических средств и переустановки программных средств.

\subsubsection{Требования к эргономике и технической эстетике}

Взаимодействие пользователя с программой должно осуществляться посредством визуального графического интерфейса. Использование манипулятора типа <<мышь>>, клавиатуры. Элементы интерфейса должны быть схожими со стандартными элементами интерфейса приложений операционной системы. 

\subsubsection{Требования к эксплуатации}

Требования к помещениям, где эксплуатируются компьютеры, на которых предполагается работа системы, к микроклимату, акустическим шумам и вибрациям, освещению, организации и оборудованию рабочих мест определены санитарно-эпидемиологическими правилами и нормативами СанПиН 2.2.2/2.4.1340-03 <<Гигиенические требования к персональным электронно-вычислительным машинам и организации работы>>.

\subsubsection{Требования к маркировке и упаковке}

Программа поставляется в виде программного изделия на внешнем оптическом носителе (CD). 
Упаковка должна обеспечить защиту носителя от пыли и небольших механических повреждений (царапин). Упаковкой служит жесткий футляр с указанием названия программного продукта.

\subsection{Требования к функциям (задачам), выполняемым системой}
\subsubsection{Требования к составу выполняемых функций}

Программа должна считывать вводимые пользователям значения, выводить результаты запросов и генерировать талоны в формате PDF. При ошибочных действиях пользователя, связанных с вводом данных или ошибками при обращении к БД, программа должна уведомлять пользователя об ошибке.

\subsection{Требования к видам обеспечения}
\subsubsection{Требования к техническому обеспечению системы}

Для работы системы требуется компьютер, на котором функционирует операционная система OC Linux Ubuntu/Debian. Для работы программы необходимы:

\begin{itemize}
  \item процессор x86-архитектуры;
  \item объем ОЗУ для выполнения программы: не менее 300 Мб;
  \item объём видеопамяти: не менее 300 Мб;
  \item память на жестком диске: не менее 100 Мб для файлов БД и программы;
  \item монитор с разрешением 800x600 или выше;
  \item мышь, клавиатура;
  \item устройство для чтения CD;
  \item принтер.
\end{itemize}

\subsubsection{Требования к информационной и программной совместимости}

Работа приложения гарантируется в OC Linux Ubuntu/Debian.

