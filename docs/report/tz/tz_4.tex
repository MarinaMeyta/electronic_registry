\subsection{Общие требования к системе}
\subsubsection{Входные и выходные данные}

Входные данные:

В программе <<hospital\_register>> возможны следующие входные данные:
Ф.И.О. сотрудника, специальность сотрудника, день записи, серия и номер паспорта пациента, номер кабинета, день недели, ID сотрудника, начало и конец смены сотрудника, ID пациента, ID расписания и ID талона, Ф.И.О. пациента, день рождения пациента, месяц рождения пациента, год рождения пациента, пол пациента, код подразделения, адрес места жительства пациента, серия полиса пациента, номер полиса пациента, страховая компания.
Далее приведен диапазон допустимых значений. Все входные значения не должны быть пустыми (NULL).

В главном окне программы MainWindow возможен только один входной параметр --- пароль администратора. На него заведомо накладывается только одно ограничение --- пароль не должен быть пустой строкой.

В окне управления базой AdminWindow входные параметры следующие: Ф.И.О. сотрудника, специальность сотрудника, номер кабинета, день недели, ID сотрудника, начало и конец смены сотрудника, ID пациента, ID расписания и ID талона (таблица~\ref{tab:tab_1}). 

\begin{table}[ht]
\caption{Таблица признаков для AdminWindow}
\label{tab:tab_1}
\begin{center}
\begin{tabularx}{\linewidth}{|X|X|X|}
\hline
 Наименование признака & Описание признака & Диапазон допустимых значений\\
\hline
 employee\_name & Ф.И.О. сотрудника & Символы русского или латинского алфавита\\
\hline
 speciality & Специальность сотрудника & Символы русского или латинского алфавита\\
\hline
 office\_number & Номер кабинета сотрудника & 3 цифры от 0 до 9\\
\hline
 week\_day & День недели & Одно из заранее определенных значений (<<Пн>>, <<Вт>>, <<Ср>>, <<Чт>>, <<Пт>>, <<Сб>>, <<Вс>>)\\
\hline
 employee\_id & Идентификатор сотрудника & 10 цифр от 0 до 9 \\
\hline
 patient\_id & Идентификатор пациента & 10 цифр от 0 до 9 \\
\hline
 timetable\_id & Идентификатор расписания & 10 цифр от 0 до 9 \\
\hline
 talon\_id & Идентификатор талона & 10 цифр от 0 до 9 \\
\hline
 shift\_begining & Начало смены сотрудника & Строка формата <<ЧЧ:ММ>>, где ЧЧ (часы) и ММ (минуты) --- цифры от 0 до 9 \\
\hline
 shift\_ending & Конец смены сотрудника & Строка формата <<ЧЧ:ММ>>, где ЧЧ (часы) и ММ (минуты) --- цифры от 0 до 9\\
\hline
\end{tabularx}
\end{center}
\end{table}

В окне для записи на прием к врачу EnrollWindow входные параметры: Ф.И.О. сотрудника, специальность сотрудника, день записи, серия и номер паспорта пациента (таблица~\ref{tab:tab_2}). 

\begin{table}[ht]
\caption{Таблица признаков для EnrollWindow}
\label{tab:tab_2}
\begin{center}
\begin{tabularx}{\linewidth}{|X|X|X|}
\hline
 Наименование признака & Описание признака & Диапазон допустимых значений\\
\hline
 employee\_name & Ф.И.О. сотрудника & Одно из заранее определенных значений\\
\hline
 speciality & Специальность сотрудника & Одно из заранее определенных значений\\
\hline
 week\_day & День записи & Одно из заранее определенных значений (<<Пн>>, <<Вт>>, <<Ср>>, <<Чт>>, <<Пт>>, <<Сб>>, <<Вс>>)\\
\hline
 passport & Серия пасспорта пациента & 4 цифры от 0 до 9 \\
\hline
 passport & Номер паспорта пациента & 6 цифр от 0 до 9 \\
\hline
\end{tabularx}
\end{center}
\end{table}

В окне регистрации пациента PatientRegisterWindow входные параметры: Ф.И.О. пациента, день рождения пациента, месяц рождения пациента, год рождения пациента, пол пациента, серия паспорта пациента, номер паспорта пациента, код подразделения, адрес места жительства пациента, серия полиса пациента, номер полиса пациента, страховая компания (таблица~\ref{tab:tab_3}). 

\begin{table}[ht]
\caption{Таблица признаков для EnrollWindow}
\label{tab:tab_3}
\begin{center}
\begin{tabularx}{\linewidth}{|X|X|X|}
\hline
 Наименование признака & Описание признака & Диапазон допустимых значений\\
\hline
 patient\_name & Ф.И.О. сотрудника & Символы русского или латинского алфавита\\
\hline
 birth\_date & День рождения пациента & Число от 1 до 31\\
\hline
 birth\_date & Месяц рождения пациента & Одно из заранее определенных значений\\
\hline
 birth\_date & Год рождения пациента & Число от 1900 до 2020\\
\hline
 sex & Пол пациента & Одно из заранее определенных значений (<<М>> или <<Ж>>)\\
\hline
 passport & Серия пасспорта пациента & 4 цифры от 0 до 9 \\
\hline
 passport & Номер паспорта пациента & 6 цифр от 0 до 9 \\
\hline
 issue\_date & День выдачи паспорта & Число от 1 до 31\\
\hline
 issue\_date & Месяц выдачи паспорта & Одно из заранее определенных значений\\
\hline
 issue\_date & Год выдачи паспорта & Число от 1900 до 2020\\
\hline
 policy & Серия полиса пациента & 5 цифр от 0 до 9 \\
\hline
 policy & Номер полиса пациента & 6 цифр от 0 до 9 \\
\hline
 insurance\_agency & Название страховой медицинской компании & Символы русского или латинского алфавита \\
\hline
\end{tabularx}
\end{center}
\end{table}

\clearpagepage


Выходные данные:

Выходные данные сохраняются в PDF-файл <<talon.pdf>> в рабочей директории программы. Данный файл в дальнейшем может быть распечатан терминалом при завершении операции записи пациента. 

В файл <<talon.pdf>> выводятся следующие данные: Ф.И.О. врача, специальность, номер кабинета, рабочие часы, дата и день недели, Ф.И.О. пациента и страховой медицинский полис пациента.


\clearpage
\subsubsection{Требования к транспортированию и хранению}

Программа поставляется на CD, который должен быть помещен в жесткий футляр, обеспечивающий длительное хранение в отапливаемых помещениях в условиях, пригодных для хранения оптических дисков.
При транспортировании и хранении программного изделия должна быть предусмотрена защита от попадания пыли и атмосферных осадков. Климатические условия транспортирования:

\begin{itemize}
  \item температура окружающего воздуха, °С – от 5 до 50;
  \item атмосферное давление, кПа – 84,0-106,7;
  \item относительная влажность воздуха при 25 °С – 45-80\%.
\end{itemize}

\subsubsection{Требования к информационной и программной совместимости}

Программное обеспечение <<hospital\_register>> с ОС Linux Ubuntu/Debian.

Минимальные системные требования:

\begin{itemize}
  \item ОС Linux Ubuntu/Debian;
  \item процессор x86-архитектуры;
  \item объем ОЗУ для выполнения программы: не менее 300 Мб;
  \item объём видеопамяти: не менее 300 Мб;
  \item память на жестком диске: не менее 100 Мб для файлов БД и программы;
  \item монитор с разрешением 800x600 или выше;
  \item мышь, клавиатура;
  \item устройство для чтения CD;
  \item принтер.
\end{itemize}

\subsubsection{Требования к надежности}

При условии соблюдения требований эксплуатации и при контроле входных и выходных данных программа должна безотказно выполнять функции, определенные настоящим техническим заданием.

Время восстановления после отказа, вызванного сбоем технических средств, сбоем операционной системы, не должно превышать времени, необходимого на перезагрузку операционной системы и запуск программы.

Время восстановления после отказа, вызванного неисправностью технических средств, жестким сбоем операционной системы, не должно превышать времени, требуемого на устранение неисправностей технических средств и переустановки программных средств.

\subsubsection{Требования к эргономике и технической эстетике}

Взаимодействие пользователя с программой должно осуществляться посредством визуального графического интерфейса. Использование манипулятора типа <<мышь>>, клавиатуры. Элементы интерфейса должны быть схожими со стандартными элементами интерфейса приложений операционной системы. 

\subsubsection{Требования к эксплуатации}

Требования к помещениям, где эксплуатируются компьютеры, на которых предполагается работа системы, к микроклимату, акустическим шумам и вибрациям, освещению, организации и оборудованию рабочих мест определены санитарно-эпидемиологическими правилами и нормативами СанПиН 2.2.2/2.4.1340-03 <<Гигиенические требования к персональным электронно-вычислительным машинам и организации работы>>.

\subsubsection{Требования к маркировке и упаковке}

Программа поставляется в виде программного изделия на внешнем оптическом носителе (CD). 
Упаковка должна обеспечить защиту носителя от пыли и небольших механических повреждений (царапин). Упаковкой служит жесткий футляр с указанием названия программного продукта.

\subsection{Требования к функциям (задачам), выполняемым системой}
\subsubsection{Требования к составу выполняемых функций}

Программа должна считывать вводимые пользователям значения, выводить результаты запросов и генерировать талоны в формате PDF. При ошибочных действиях пользователя, связанных с вводом данных или ошибками при обращении к БД, программа должна уведомлять пользователя об ошибке.

\subsection{Требования к видам обеспечения}
\subsubsection{Требования к техническому обеспечению системы}

Для работы системы требуется компьютер, на котором функционирует операционная система OC Linux Ubuntu/Debian. Для работы программы необходимы:

\begin{itemize}
  \item процессор x86-архитектуры;
  \item объем ОЗУ для выполнения программы: не менее 300 Мб;
  \item объём видеопамяти: не менее 300 Мб;
  \item память на жестком диске: не менее 100 Мб для файлов БД и программы;
  \item монитор с разрешением 800x600 или выше;
  \item мышь, клавиатура;
  \item устройство для чтения CD;
  \item принтер.
\end{itemize}

\subsubsection{Требования к информационной и программной совместимости}

Работа приложения гарантируется в OC Linux Ubuntu/Debian.

