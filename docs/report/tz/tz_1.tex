\subsection{Полное наименование системы и ее условное обозначение}

Полное название программы: <<Электронная регистратура>>.

Условное обозначение: <<hospital\_register>>.


\subsection{Наименование предприятий (объединений) разработчика и заказчика (пользователя) системы и их реквизиты}

Разработчик: студентка гр.722 ФБ ТУСУРа: Мейта Марина Валерьевна.

Заказчик: Томский государственный университет систем управления и радиоэлектроники (ТУСУР), факультет безопасности (ФБ), в лице аспиранта кафедры комплексной информационной безопасности электронно-вычислительных систем (КИБЭВС) Горбунова И. В.

\subsection{Требования, на основании которых создается система, и даты их утверждения}

Задание на выполнение курсового проекта по дисциплине <<Безопасность систем баз данных>> утверждено Горбуновым И. В. 1 марта 2015 г.

\subsection{Плановые сроки начала и окончания работы по созданию системы}

Дата начала работы --- 1 марта 2015 года, дата окончания работы --- 1 июня 2015 года. 

\subsection{Сведения об источниках и порядке финансирования работ}

Финансирование осуществляется лицами, заинтересованными в разработке программного средства, а именно разработчиком из собственных средств.

\subsection{Прядок оформления и предъявления заказчику результатов работ по созданию системы (ее частей), по изготовлению и наладке отдельных средств (технических, программных, информационных) и программно-технических (программно-методических) комплексов системы}

Предоставляется промежуточная отчетность по завершении каждого установленного заказчиком этапа разработки. 
Документы предъявляются на бумажных носителях и в электронном виде не позднее установленных сроков. Этапы и сроки сдачи отчетности приведены в таблице~\ref{tab_1:tab_1}.

\begin{table}[ht]
\caption{Этапы разработки}
\label{tab_1:tab_1}
\begin{center}
\begin{tabularx}{\linewidth}{|X|X|X|}
\hline
Содержание этапа & Сроки & Отчетный документ \\
\hline
Подготовительный этап. Постановка задачи, сбор и анализ требований к разработке, проработка прототипа ПО, проработка прототипа БД.Разработка технического задания. & 1.03 --- 21.03 & Техническое задание. Прототипы ПО и БД. \\
\hline
Проектирование & 21.03 --- 14.04 & Технический проект. Пересмотренные прототипы. ПО и БД.  \\
\hline
Реализация спроектированного приложения и базы данных. Написание программной справки. Тестирование. & 14.04 --- 02.05 & Версия программного продукта. \\
\hline
Определение соответствия, разработанного ПО заданным критериям качества. & 02.05 --- 21.05 & Версия программного продукта.Результаты исследований. Результаты тестирования.\\
\hline
Оформление пояснительной записки. Прием работы. & 21.05 --- 01.05 & Пояснительная записка. \\
\hline
\end{tabularx}
\end{center}
\end{table}


