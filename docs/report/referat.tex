\newpage
\ESKDthisStyle{empty}
\paragraph{\hfill РЕФЕРАТ \hfill}
Курсовая работа содержит \ESKDtotal{page} страниц, \ESKDtotal{figure} рисунков, \ESKDtotal{table} таблиц, \ESKDtotal{bibitem} источника, \ESKDtotal{appendix} приложение.

БАЗЫ ДАННЫХ, SQLITE, MONODEVELOP, C\#, GTKSharp.

Цель работы --- проектирование, разработка базы данных и клиентской части программного обеспечения для электронной регистрации на прием к врачу (электронная регистратура).

Результатом выполнения работы является база данных и графическое приложение для осуществления регистрации пациентов, записи на прием к специалисту, администрирования записей в базе данных (просмотр, удаление и добавление информации о сотрудниках поликлиники, пациентах, выданных талонах и др.).

В процессе работы были выполнены все вышепоставленные цели, разработана инфологическая модель данных для описания процесса регистрации и структуры базы данных, предусмотрены ограничения на ввод данных, а также применены средства обеспечения безопасности базы данных на уровне приложения.

Проект выполнен с использованием следующих средств разработки:
\begin{itemize}
  \item ОС Linux Ubuntu 14.10;
  \item язык программирования C\#;
  \item среда разработки MonoDevelop 4.0.12 \cite{monodoc};
  \item встраиваемая реляционная база данных SQLite \cite{sqlitedoc};
  \item СУБД SQLiteman 1.2.2;
  \item кроссплатформенная библиотека элементов графического интерфейса GTKSharp;
  \item iTextSharp --- .NET PDF библиотека для генерации PDF-документов \cite{itextsharp};
  \item система контроля версий Git.
\end{itemize}

Пояснительная записка выполнена при помощи системы компьютерной вёрстки \LaTeX.
