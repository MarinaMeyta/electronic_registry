SQLite является встраиваемой реляционной базой данных и не использует парадигму <<клиент-сервер>>, то есть движок SQLite не является отдельно работающим процессом, с которым взаимодействует программа, а предоставляет библиотеку, с которой программа компонуется и движок становится составной частью программы. Таким образом, в качестве протокола обмена используются вызовы функций (API) библиотеки SQLite. Такой подход уменьшает накладные расходы, время отклика и упрощает программу. SQLite хранит всю базу данных (включая определения, таблицы, индексы и данные) в единственном стандартном файле на том компьютере, на котором исполняется программа.\cite{sqlitedoc}

В SQLite отсутствует разграничение ролей как таковых, поэтому потребовалось вводить меры по защите БД непосредственно в программном приложении для работы с базой. На уровне приложения были условно созданы 2 роли: <<администратор>> и <<пациент>>. <<Администратор>> при вводе пароля (хэш от пароля сравнивается с хэшем, хранимым в БД) получает доступ к программному изменению и просмотру данных в базе. <<Пациент>> имеет свободный доступ к форме для записи на прием и регистрации.
